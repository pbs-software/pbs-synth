%\newpage
%\subsubsubsection{Figures -- Simulation de r\'{e}f\'{e}rence (m\'{e}thode MCCM)}

\onefig{traceParams}{trac\'{e}s MCCM pour les param\`{e}tres estim\'{e}s. Les lignes grises montrent les \Nmcmc~ \'{e}chantillons pour chaque param\`{e}tre, les lignes pleines repr\'{e}sentent la m\'{e}diane cumulative (jusqu'\`{a} l'\'{e}chantillon en question) et les lignes tiret\'{e}es indiquent les quantiles cumulatifs 0,05 et 0,95. Les cercles rouges sont les estimations du MDP. Pour les param\`{e}tres autres que \code{delta1}, num\'{e}ros (1, 3-5) correspondent aux flottes (p\^{e}ches et \'{e}tudes).}{SCA~CB : }{car.}

\onefig{splitChain}{trac\'{e}s diagnostiques obtenus en divisant la cha\^{i}ne MCCM de \Nmcmc~\'{e}chantillons MCCM en trois segments et en superposant les distributions cumulatives du premier segment (en rouge), du deuxi\`{e}me segment (en bleu) et du dernier segment (en noir).}{SCA~CB : }{car.}

\clearpage

\onefig{paramACFs}{trac\'{e}s d'autocorr\'{e}lation pour les param\`{e}tres estim\'{e}s provenant des r\'{e}sultats MCCM. Les lignes bleues horizontales tiret\'{e}es d\'{e}limitent l'intervalle de confiance \`{a} 95\pc{} pour l'ensemble de corr\'{e}lations d\'{e}cal\'{e}es de chaque param\`{e}tre.}{SCA~CB : }{car.}

\onefig{pdfParameters}{distribution a posteriori (barres vertes verticales), profil de vraisemblance (courbe bleue fine) et fonction de densit\'{e} a priori (courbe noire \'{e}paisse) pour les param\`{e}tres estim\'{e}s. La ligne verticale tiret\'{e}e indique la m\'{e}diane a posteriori MCCM; la ligne verticale bleue repr\'{e}sente le MDP; le triangle rouge indique la valeur initiale de chaque param\`{e}tre.}{SCA~CB : }{car.}

\clearpage


%%==============================================================================
