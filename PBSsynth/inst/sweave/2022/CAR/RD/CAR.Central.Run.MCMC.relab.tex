%%\section{MCMC Results -- Tables}

\onefig{traceParams}{MCMC traces for the estimated parameters. Grey lines show the \Nmcmc~samples for each parameter, solid lines show the cumulative median (up to that sample), and dashed lines show the cumulative 0.05 and 0.95 quantiles.  Red circles are the MPD estimates. For parameters other than \code{delta1}, numbers (1, 3-5) correspond to fleets (fisheries and surveys).}{CAR~BC: }{car.}
\onefig{splitChain}{diagnostic plot obtained by dividing the MCMC chain of \Nmcmc~MCMC samples into three segments, and overplotting the cumulative distributions of the first segment (red), second segment (blue) and final segment (black).}{CAR~BC: }{car.}
\clearpage

\onefig{paramACFs}{autocorrelation plots for the estimated parameters from the MCMC output. Horizontal dashed blue lines delimit the 95\pc{} confidence interval for each parameter's set of lagged correlations.}{CAR~BC: }{car.}
\onefig{pdfParameters}{posterior distribution (vertical green bars), likelihood profile (thin blue curve), and prior density function (thick black curve) for estimated parameters. Vertical dashed line indicates the MCMC posterior median; vertical blue line represents the MPD; red triangle indicates initial value for each parameter.}{CAR~BC: }{car.}
\clearpage

%%N/A: Figures not used:
%%\onefig{pairsPars}{Kernel density plot of \Nmcmc~MCMC samples for 20 parameters. Numbers in the lower panels are the absolute values of the correlation coefficients.}{CAR~BC: }{car.}
%%\onefig{traceBiomass}{MCMC traces for female spawning biomass estimates at five-year intervals.  Note that vertical scales are different for each plot (to show convergence of the MCMC chain, rather than absolute differences in annual values). Grey lines show the \Nmcmc~samples for each parameter, solid lines show the cumulative  median (up to that sample), and dashed lines show the cumulative 0.05 and 0.95 quantiles.  Red circles are the MPD estimates.}{CAR~BC: }{car.}
%%\onefig{traceRecruits}{MCMC traces for recruitment estimates at five-year intervals. Note that vertical scales are different for each plot (to show convergence of the MCMC chain, rather than absolute differences in annual recruitment). Grey lines show the \Nmcmc~samples for each parameter, solid lines show the cumulative  median (up to that sample), and dashed lines show the cumulative 0.05 and 0.95 quantiles.  Red circles are the MPD estimates.}{CAR~BC: }{car.}
%%\clearpage
%%\twofig{boverbmsyMCMC}{depleteMCMC}{Top: estimated spawning biomass $B_t$ relative to spawning biomass at maximum sustainable yield ($\Bmsy$) (boxplots). The median biomass trajectory appears as a solid curve surrounded by a 90\pc{} credibility envelope (quantiles: 0.05-0.95) in light blue and delimited by dashed lines for years $t$=1935-2023; projected biomass appears in light red for years $t$=2024-2033. Also delimited is the 50\pc{} credibility interval (quantiles: 0.25-0.75) delimited by dotted lines. The horizontal dashed lines show the median LRP and USR. Bottom: marginal posterior distribution of depletion ($B_t/B_0$), where $t$=1935-2023.}{CAR~BC: }{car.}
%%\onefig{snail}{Phase plot through time of the medians of the ratios $B_t/B_\text{MSY}$ (the spawning biomass in year $t$ relative to $B_\text{MSY}$) and $u_{t-1} / u_\text{MSY}$ (the exploitation rate in year $t-1$ relative to $u_\text{MSY}$). The filled green circle is the starting year (1936). Years then proceed from light grey through to dark grey with the final year (2023) as a filled cyan circle, and the blue lines represent the 0.05 and 0.95 quantiles of the posterior distributions for the final year. The filled gold circles indicate the status in 1999, 2005, 2007, and 2009, which coincide with previous assessments for this species.  Red and green vertical dashed lines indicate the Precautionary Approach provisional limit and upper stock reference points (0.4, 0.8 $\Bmsy$), and the horizontal grey dotted line indicates $u$ at MSY.}{CAR~BC: }{car.}
%%\clearpage
%%\twofig{sbiomassMCMC}{sprMCMC}{Marginal posterior distribution of spawning biomass (top) and spawners-per recruit (bottom) over time. Boxplots show the 0.05, 0.25, 0.5, 0.75, and 0.95 quantiles from the MCMC results.}{CAR~BC: }{car.}
%%\twofig{recruitsMCMC}{recdevMCMC}{Marginal posterior distribution of recruitment in 1,000s of age-0 fish (top) and recruitment deviations (bottom) over time. Boxplots show the 0.05, 0.25, 0.5, 0.75, and 0.95 quantiles from the MCMC results.}{CAR~BC: }{car.}
%%\twofig{fishmortMCMC}{exploitMCMC}{Marginal posterior distribution of fishing mortality (top) and exploitation rate (bottom) over time. Boxplots show the 0.05, 0.25, 0.5, 0.75, and 0.95 quantiles from the MCMC results.}{CAR~BC: }{car.}
%%\clearpage

%%==============================================================================
