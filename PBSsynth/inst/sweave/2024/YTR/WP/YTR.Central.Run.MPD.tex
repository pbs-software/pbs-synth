%% Modified from MPD.21.01.v3.tex
%%---------------------

%%..............................................................................
%\newpage
\subsubsubsection{MPD tables}

%%--Table 1--------------------
%\begin{table}[h!]
%\centering
%\caption{Estimated biomass, spawning ($B$) and total ($T$), in 2025 and relative to virgin biomass ($B_0$, $T_0$) in 1935.}
%\label{tab:biomass}
%\begin{tabular}{lrrr} 
%\hline \\ [-1.5ex]
%{\bf Biomass} & {\bf $B_{t=2025}$} & {\bf $B_0$} & {\bf $B_{2025}$~/~$B_0$} \\ [1ex]
%\hline \\ [-1.5ex]
%Spawning  & 10,874 & 18,816 & 0.578\\Vulnerable  & 48,229 & 87,921 & 0.549\\Total  & 73,156 & 117,537 & 0.622\\%Spawning & 10,874 & 18,816 & 0.578 \\
%%Total    & 73,156 & 117,537 & 0.622 \\
%\\ [-1.5ex]
%\hline
%\end{tabular}
%\end{table}

%\clearpage
%\qquad % or \hspace{2em}


%%--Table 2--------------------
\setlength{\tabcolsep}{2pt}
\begin{table}[!h]
\centering
\caption{Priors and MPD estimates for estimated parameters. Prior information -- distributions: 0~=~uniform, 2~=~beta, 6~=~normal}
\label{tab:parest}
\usefont{\encodingdefault}{\familydefault}{\seriesdefault}{\shapedefault}\small
\begin{tabular}{lcccccr}
\hline \\ [-1.5ex]
%\multicolumn{6}{l}{{\bf Parameter in write-up, Awatea input name, Awatea export name}} \\
{\bf Parameter} & {\bf Phase} & {\bf Range} & {\bf Type} & {\bf (Mean,SD)} & {\bf Initial} & {\bf MPD} \\ [1ex]
\hline \\ [-1.5ex]
$M_1$ (female) & 4 & (0.02, 0.2) & 6 & (0.08, 0.024) & 0.08 & 0.125 \\
$M_2$ (male)   & 4 & (0.02, 0.2) & 6 & (0.08, 0.024) & 0.08 & 0.102 \\
$\log R_0$   & 1 & (1, 16) & 6 & (10, 10) & 10 & 9.713 \\
BH $h$ & 5 & (0.2, 1) & 2 & (0.67, 0.17) & 0.67 & 0.810 \\
$\mu_1$ (TRAWL BC) & 3 & (5, 40) & 6 & (11, 11) & 11 & 10.418 \\
$\log v_{\text{L}1}$ (TRAWL BC) & 4 & (-15, 15) & 6 & (2, 2) & 2 & 1.692 \\
$\Delta_1$ (TRAWL BC) & 4 & (-10, 10) & 6 & (0, 10) & 0 & 0.136 \\
$\mu_2$ (QCS) & 3 & (5, 40) & 6 & (11, 11) & 11 & 9.552 \\
$\log v_{\text{L}2}$ (QCS) & 4 & (-15, 15) & 6 & (2, 2) & 2 & 1.042 \\
$\Delta_2$ (QCS) & 4 & (-10, 10) & 6 & (0, 10) & 0 & 0.153 \\
$\mu_3$ (WCVI) & 3 & (5, 40) & 6 & (11, 11) & 11 & 11.944 \\
$\log v_{\text{L}3}$ (WCVI) & 4 & (-15, 15) & 6 & (2, 2) & 2 & 3.099 \\
$\Delta_3$ (WCVI) & 4 & (-10, 10) & 6 & (0, 10) & 0 & 0.341 \\
$\mu_4$ (WCHG) & -3 & (5, 40) & 0 & (--, --) & 15 & 15 \\
$\log v_{\text{L}4}$ (WCHG) & -4 & (-15, 15) & 0 & (--, --) & 3 & 3 \\
$\Delta_4$ (WCHG) & -4 & (-10, 10) & 0 & (--, --) & 0 & 0 \\
$\mu_5$ (HS) & -3 & (5, 40) & 0 & (--, --) & 10 & 10 \\
$\log v_{\text{L}5}$ (HS) & -4 & (-15, 15) & 0 & (--, --) & 2 & 2 \\
$\Delta_5$ (HS) & -4 & (-10, 10) & 0 & (--, --) & 0 & 0 \\
$\mu_7$ (NMFS) & 3 & (1, 40) & 6 & (11, 11) & 11 & 10.712 \\
$\log v_{\text{L}7}$ (NMFS) & 4 & (-15, 15) & 6 & (2, 2) & 2 & 2.061 \\
$\Delta_7$ (NMFS) & 4 & (-8, 10) & 6 & (0, 10) & 0 & 0.220 \\
\hline
\end{tabular}
\usefont{\encodingdefault}{\familydefault}{\seriesdefault}{\shapedefault}\normalsize
\end{table}

\clearpage
%\qquad % or \hspace{2em}

%%--Tables 3-5 -----------------
%% Likelihoods Used from replist
%% Get numbers from chunk above
\setlength{\tabcolsep}{0pt}
\begin{longtable}[c]{>{\raggedright\let\newline\\\arraybackslash\hspace{0pt}}p{2.31in}>{\raggedleft\let\newline\\\arraybackslash\hspace{0pt}}p{1.35in}>{\raggedleft\let\newline\\\arraybackslash\hspace{0pt}}p{1.35in}}
  \caption{Likelihood components reported in \texttt{likelihoods\_used}.} \label{tab:like1}\\  \hline\\[-2.2ex]
  Likelihood Component  & values & lambdas \\[0.2ex]\hline\\[-1.5ex]  \endfirsthead   \hline  
  Likelihood Component  & values & lambdas \\[0.2ex]\hline\\[-1.5ex]  \endhead  \hline\\[-2.2ex]   \endfoot  \hline \endlastfoot
  TOTAL & 384.3 & -- \\ 
  %Catch & 0 & -- \\ 
  Equilibrium catch & 0 & -- \\ 
  Survey & 39.81 & -- \\ 
  Age composition& 332.7 & -- \\ 
  Recruitment & 9.120 & 1 \\ 
  Initial equilibrium regime & 0 & 1 \\ 
  Forecast recruitment & 0.1144 & 1 \\ 
  Parameter priors & 2.542 & 1 \\ 
  Parameter softbounds & 0.002270 & -- \\ 
  Parameter deviations & 0 & 1 \\ 
  Crash penalty & 0 & 1 \\ 
   %\hline
\end{longtable}\setlength{\tabcolsep}{0pt}

%\begin{longtable}[c]{>{\raggedright\let\newline\\\arraybackslash\hspace{0pt}}p{2.6in}>{\raggedleft\let\newline\\\arraybackslash\hspace{0pt}}p{1.2in}>{\raggedleft\let\newline\\\arraybackslash\hspace{0pt}}p{1.2in}}
%  \caption{Likelihood components reported in \texttt{likelihoods\_laplace}.} \label{tab:like2}\\  \hline\\[-2.2ex]  Likelihood Component  & values & lambdas \\[0.2ex]\hline\\[-1.5ex]  \endfirsthead   \hline  Likelihood Component  & values & lambdas \\[0.2ex]\hline\\[-1.5ex]  \endhead  \hline\\[-2.2ex]   \endfoot  \hline \endlastfoot  NoBias corr Recruitment(info only) & 2.567 & 1 \\ 
%  Laplace obj fun(info only) & 377.7 & -- \\ 
%   %\hline
%\end{longtable}\usefont{\encodingdefault}{\familydefault}{\seriesdefault}{\shapedefault}\small \setlength{\tabcolsep}{0pt}

\begin{longtable}[c]{>{\raggedleft\let\newline\\\arraybackslash\hspace{0pt}}p{0.8in}>{\raggedleft\let\newline\\\arraybackslash\hspace{0pt}}p{0.61in}>{\raggedleft\let\newline\\\arraybackslash\hspace{0pt}}p{0.78in}>{\raggedleft\let\newline\\\arraybackslash\hspace{0pt}}p{0.61in}>{\raggedleft\let\newline\\\arraybackslash\hspace{0pt}}p{0.64in}>{\raggedleft\let\newline\\\arraybackslash\hspace{0pt}}p{0.64in}>{\raggedleft\let\newline\\\arraybackslash\hspace{0pt}}p{0.61in}>{\raggedleft\let\newline\\\arraybackslash\hspace{0pt}}p{0.68in}>{\raggedleft\let\newline\\\arraybackslash\hspace{0pt}}p{0.68in}}
  \caption{Likelihood components reported in \texttt{likelihoods\_by\_fleet}. Notation: $\lambda$~= emphasis factors in the likelihood; $\Lagr$~= negative log likelihood} \label{tab:like3}\\  \hline\\[-2.2ex]  
  Label  & ALL & BC\newline TRAWL & QCS\newline SYN & WCVI\newline SYN & WCHG\newline SYN & HS\newline SYN & GIG\newline HIS & NMFS\newline TRI \\[0.2ex]\hline\\[-1.5ex]  \endfirsthead   \hline  
  Label  & ALL & BC\newline TRAWL & QCS\newline SYN & WCVI\newline SYN & WCHG\newline SYN & HS\newline SYN & GIG\newline HIS & NMFS\newline TRI \\[0.2ex]\hline\\[-1.5ex]  \endhead  \hline\\[-2.2ex]   \endfoot  \hline \endlastfoot  
  Catch $\lambda$      & -- & 1.000 & 1.000 & 1.000 & 1.000 & 1.000 & 1.000 & 1.000 \\ 
  Catch $\Lagr$        & 0 & 0 & 0 & 0 & 0 & 0 & 0 & 0 \\ 
  Initial EQ $\lambda$ & -- & 1.000 & 1.000 & 1.000 & 1.000 & 1.000 & 1.000 & 1.000 \\ 
  Initial EQ $\Lagr$   & 0 & 0 & 0 & 0 & 0 & 0 & 0 & 0 \\ 
  Survey $\lambda$     & -- & 0 & 1.000 & 1.000 & 1.000 & 1.000 & 1.000 & 1.000 \\ 
  Survey $\Lagr$       & 39.81 & 0 & -4.583 & 16.69 & 8.304 & 3.105 & 6.250 & 10.04 \\ 
  Surv N use           & -- & 0 & 12.00 & 10.00 & 10.00 & 10.00 & 8.000 & 7.000 \\ 
  Surv N skip          & -- & 0 & 0 & 0 & 0 & 0 & 0 & 0 \\ 
  Age $\lambda$        & -- & 1.000 & 1.000 & 1.000 & 0 & 0 & 0 & 1.000 \\ 
  Age $\Lagr$          & 332.7 & 219.8 & 41.56 & 46.89 & 0 & 0 & 0 & 24.42 \\ 
  Age N use            & -- & 38.00 & 11.00 & 7.000 & 0 & 0 & 0 & 6.000 \\ 
  Age N skip           & -- & 0 & 0 & 0 & 0 & 0 & 0 & 0 \\ 
   %\hline
\end{longtable}\usefont{\encodingdefault}{\familydefault}{\seriesdefault}{\shapedefault}\normalsize \clearpage

%%--Table 6 -----------------
%% Residuals -- sum of Studentised residuals by fleet 
%\setlength{\tabcolsep}{0pt}
%\begin{longtable}[c]{>{\raggedleft\let\newline\\\arraybackslash\hspace{0pt}}p{1.16in}>{\raggedleft\let\newline\\\arraybackslash\hspace{0pt}}p{1.52in}>{\raggedleft\let\newline\\\arraybackslash\hspace{0pt}}p{1.16in}>{\raggedleft\let\newline\\\arraybackslash\hspace{0pt}}p{1.16in}}
%  \caption{Sum of residuals for observed and expected mean ages by fleet.} \label{tab:ssr}\\  \hline\\[-2.2ex]  Run Rwt Ver  & Fleet & Sum Std Res & Sum PJS Res \\[0.2ex]\hline\\[-1.5ex]  \endfirsthead   \hline  Run Rwt Ver  & Fleet & Sum Std Res & Sum PJS Res \\[0.2ex]\hline\\[-1.5ex]  \endhead  \hline\\[-2.2ex]   \endfoot  \hline \endlastfoot  R.02.01.v1 & BC Trawl Fishery & -7.836 & 39.587 \\ 
%  R.02.01.v1 & QCS Synoptic & 3.483 & 16.320 \\ 
%  R.02.01.v1 & WCVI Synoptic & 2.223 & 7.997 \\ 
%  R.02.01.v1 & NMFS Triennial & -2.014 & 9.379 \\ 
%  R.02.01.v1 & Total & -4.144 & 73.283 \\ 
%   %\hline
%\end{longtable}
\clearpage

%%..............................................................................
\subsubsubsection{MPD figures}

\onefig{mleParameters}{Likelihood profiles (thin blue curves) and prior density functions (thick black curves) for the estimated parameters. Vertical lines represent the maximum likelihood estimates; red triangles indicate initial values used in the minimization process.}
\onefig{survIndSer}{Survey index values (points) with 95\pc{} confidence intervals (bars) and MPD model fits (curves) for the fishery-independent survey series.}
\onefig{survRes}{Survey index residuals calculated as (log(Obs) - log(Exp))/SE, where SE is the total standard error including any estimated additional uncertainty.}
\clearpage

\onefig{agefitFleet1}{BC Trawl Fishery proportions-at-age (bars=observed, lines=predicted) for females (upright) and males (inverted).}
\onefig{osa.residuals.fleet1.sexF}{BC Trawl Fishery one-step-ahead residuals for females: (top left) bubble plot of composition residuals (positive in blue, negative in red, legend at bottom left); (top right) autocorrelation by row (age), column (year), and diagonal; (bottom left) quantile-quantile plot with points colour-coded by age; (bottom right) standard normal residuals, points colour-coded by age}
\onefig{osa.residuals.fleet1.sexM}{BC Trawl Fishery one-step-ahead residuals for males: (top left) bubble plot of composition residuals (positive in blue, negative in red, legend at bottom left); (top right) autocorrelation by row (age), column (year), and diagonal; (bottom left) quantile-quantile plot with points colour-coded by age; (bottom right) standard normal residuals, points colour-coded by age}
\onefig{ageresFleet1}{BC Trawl Fishery residuals of model fits to proportion-at-age data. Vertical axes are standardised residuals. Boxplots in three panels show residuals by age class, by year of data, and by year of birth (following a cohort through time). Cohort boxes are coloured green if recruitment deviations in birth year are positive, red if negative. Boxes give quantile ranges (0.25-0.75) with horizontal lines at medians, vertical whiskers extend to the the 0.05 and 0.95 quantiles, and outliers appear as plus signs.}
\clearpage 

\onefig{agefitFleet2}{QCS Synoptic proportions-at-age (bars=observed, lines=predicted) for females (upright) and males (inverted).}
\onefig{osa.residuals.fleet2.sexF}{QCS Synoptic one-step-ahead residuals for females: (top left) bubble plot of composition residuals (positive in blue, negative in red, legend at bottom left); (top right) autocorrelation by row (age), column (year), and diagonal; (bottom left) quantile-quantile plot with points colour-coded by age; (bottom right) standard normal residuals, points colour-coded by age}
\onefig{osa.residuals.fleet2.sexM}{QCS Synoptic one-step-ahead residuals for males: (top left) bubble plot of composition residuals (positive in blue, negative in red, legend at bottom left); (top right) autocorrelation by row (age), column (year), and diagonal; (bottom left) quantile-quantile plot with points colour-coded by age; (bottom right) standard normal residuals, points colour-coded by age}
\onefig{ageresFleet2}{QCS Synoptic residuals of model fits to proportion-at-age data. Vertical axes are standardised residuals. Boxplots in three panels show residuals by age class, by year of data, and by year of birth (following a cohort through time). Cohort boxes are coloured green if recruitment deviations in birth year are positive, red if negative. Boxes give quantile ranges (0.25-0.75) with horizontal lines at medians, vertical whiskers extend to the the 0.05 and 0.95 quantiles, and outliers appear as plus signs.}
\clearpage 

\onefig{agefitFleet3}{WCVI Synoptic proportions-at-age (bars=observed, lines=predicted) for females (upright) and males (inverted).}
\onefig{osa.residuals.fleet3.sexF}{WCVI Synoptic one-step-ahead residuals for females: (top left) bubble plot of composition residuals (positive in blue, negative in red, legend at bottom left); (top right) autocorrelation by row (age), column (year), and diagonal; (bottom left) quantile-quantile plot with points colour-coded by age; (bottom right) standard normal residuals, points colour-coded by age}
\onefig{osa.residuals.fleet3.sexM}{WCVI Synoptic one-step-ahead residuals for males: (top left) bubble plot of composition residuals (positive in blue, negative in red, legend at bottom left); (top right) autocorrelation by row (age), column (year), and diagonal; (bottom left) quantile-quantile plot with points colour-coded by age; (bottom right) standard normal residuals, points colour-coded by age}
\onefig{ageresFleet3}{WCVI Synoptic residuals of model fits to proportion-at-age data. Vertical axes are standardised residuals. Boxplots in three panels show residuals by age class, by year of data, and by year of birth (following a cohort through time). Cohort boxes are coloured green if recruitment deviations in birth year are positive, red if negative. Boxes give quantile ranges (0.25-0.75) with horizontal lines at medians, vertical whiskers extend to the the 0.05 and 0.95 quantiles, and outliers appear as plus signs.}
\clearpage 

\onefig{agefitFleet7}{NMFS Triennial proportions-at-age (bars=observed, lines=predicted) for females (upright) and males (inverted).}
\onefig{osa.residuals.fleet7.sexF}{NMFS Triennial one-step-ahead residuals for females: (top left) bubble plot of composition residuals (positive in blue, negative in red, legend at bottom left); (top right) autocorrelation by row (age), column (year), and diagonal; (bottom left) quantile-quantile plot with points colour-coded by age; (bottom right) standard normal residuals, points colour-coded by age}
\onefig{osa.residuals.fleet7.sexM}{NMFS Triennial one-step-ahead residuals for males: (top left) bubble plot of composition residuals (positive in blue, negative in red, legend at bottom left); (top right) autocorrelation by row (age), column (year), and diagonal; (bottom left) quantile-quantile plot with points colour-coded by age; (bottom right) standard normal residuals, points colour-coded by age}
\onefig{ageresFleet7}{NMFS Triennial residuals of model fits to proportion-at-age data. Vertical axes are standardised residuals. Boxplots in three panels show residuals by age class, by year of data, and by year of birth (following a cohort through time). Cohort boxes are coloured green if recruitment deviations in birth year are positive, red if negative. Boxes give quantile ranges (0.25-0.75) with horizontal lines at medians, vertical whiskers extend to the the 0.05 and 0.95 quantiles, and outliers appear as plus signs.}
\clearpage 

\onefig{meanAge}{Mean ages each year for the \textbf{weighted} data (solid circles) with 95\pc{} confidence intervals and model estimates (blue lines) for the commercial and survey age data.}
\onefig{selectivity}{Selectivities for commercial fleet catch and surveys (all MPD values), with maturity ogive for females indicated by `m'.}
%\onefig{harvest}{Time series of harvest (or exploitation) rate by fishery. Bars along the bottom show catch biomass (tonnes) for all fisheries.}
\clearpage 

%\onefig{spawning}{Time series of biomass (spawning, female, male, total) in tonnes. Uncertainty envelope generated by SS is provided for spawning female biomass. Bars along the bottom show catch biomass (tonnes) for all fisheries.}
%\onefig{vulnerable}{Time series of biomass (vulnerable and total) in tonnes. Each fishery taps into a vulnerable proportion of the total population based on fleet selectivity. \emph{Note}: in SS3, \code{`sel(B)'} appears to equal the catch (\code{`obs\_cat'}) in the \code{ts} object; vulnerable biomass is calculated as catch over harvest rate (\code{V=C/u}).  Bars along the bottom show catch biomass (tonnes) for all fisheries.}
%\clearpage

%\twofig{Bt}{BtB0}{Spawning biomass $B_t$ (tonnes, mature females) over time (top), and spawning biomass $B_t$ relative to unfished equilbrium spawning biomass $B_0$ (bottom). Blue line designates SS fit for 2025.}
\twofig{BtB0}{harvest}{Spawning biomass $B_t$ relative to unfished equilbrium spawning biomass $B_0$ (top) and harvest (or exploitation) rate (bottom). Circles indicate population reconstruction (\startYear-\currYear), triangles indicate projections out to \projYear{} at \projCatch~t/y. Bars along the bottom show catch biomass (tonnes) for the BC commercial fishery.}
\twofig{recruits}{recDev}{Recruitment (thousands of fish) over time (top) and log of annual recruitment deviations (bottom), $\epsilon_t$, where bias-corrected multiplicative deviation is  $\mbox{e}^{\epsilon_t - \sigma_R^2/2}$ and  $\epsilon_t \sim \mbox{Normal}(0, \sigma_R^2)$. Blue line designates 2025 SS fit for age-0 fish.}
\clearpage
\onefig{stockRecruit}{Deterministic stock-recruit relationship (black curve) and observed values (labelled by year of spawning).}

%%==============================================================================
