%%Annexe F R\'{e}sultats -- Figures relatives au cycle central -- commence \`{a} la page 206 du document de r\'{e}f\'{e}rence

%% YMR Run 75.01 MCMC output
\clearpage

\subsubsection{Figures relatives au cycle central}

%\onefig{pairsPars}{Kernel density plot of \Nmcmc~MCMC samples for 11 parameters. Numbers in the lower panels are the absolute values of the correlation coefficients.}{Central Run 75: }{ymr.}

\onefig{traceParams}{Trac\'{e}s MCCM pour les param\`{e}tres estim\'{e}s. Les lignes grises montrent les \Nmcmc~\'{e}chantillons pour chaque param\`{e}tre, les lignes pleines la m\'{e}diane cumulative (jusqu'\`{a} cet \'{e}chantillon), et les lignes pointill\'{e}es les quantiles cumulatifs de 0,05 et 0,95. Les cercles rouges repr\'{e}sentent les estimations du MDP. Pour les param\`{e}tres autres que $M$ (s'ils sont estim\'{e}s), les chiffres 1 \`{a} 5 en indice correspondent aux flottilles dans SS (une p\^{e}che et quatre relev\'{e}s).}{Cycle central E75~: }{ymr.}

\onefig{splitChain}{Trac\'{e} diagnostique obtenu en divisant la cha\^{i}ne MCCM des \Nmcmc~MCMC \'{e}chantillons en trois segments, et en superposant les distributions cumulatives du premier segment (rouge), du deuxi\`{e}me segment (bleu) et du dernier segment (noir).}{Cycle central E75~: }{ymr.}

\clearpage

\onefig{paramACFs}{Trac\'{e}s d'autocorr\'{e}lation pour les param\`{e}tres estim\'{e}s provenant des r\'{e}sultats MCCM. Les lignes bleues pointill\'{e}es horizontales d\'{e}limitent l'intervalle de confiance \`{a} 95\pc{} pour l'ensemble des corr\'{e}lations d\'{e}cal\'{e}es de chaque param\`{e}tre.}{Cycle central E75~: }{ymr.}

\onefig{pdfParameters}{Distribution a posteriori (barres vertes verticales), profil de vraisemblance (courbe bleue fine) et fonction de densit\'{e} a priori (courbe noire \'{e}paisse) pour les param\`{e}tres estim\'{e}s. La ligne pointill\'{e}e verticale correspond \`{a} la m\'{e}diane a posteriori de la m\'{e}thode MCCM; la ligne bleue verticale repr\'{e}sente le MDP; le triangle rouge marque la valeur initiale de chaque param\`{e}tre.}{Cycle central E75~: }{ymr.}

\clearpage

%\onefig{traceBiomass}{MCMC traces for female spawning biomass estimates at five-year intervals. Note that vertical scales are different for each plot (to show convergence of the MCMC chain, rather than absolute differences in annual values). Grey lines show the \Nmcmc~samples for each parameter, solid lines show the cumulative median (up to that sample), and dashed lines show the cumulative 0.05 and 0.95 quantiles. Red circles are the MPD estimates.}{Central Run 75: }{ymr.}
%
%\onefig{traceRecruits}{MCMC traces for recruitment estimates at five-year intervals. Note that vertical scales are different for each plot (to show convergence of the MCMC chain, rather than absolute differences in annual recruitment). Grey lines show the \Nmcmc~samples for each parameter, solid lines show the cumulative median (up to that sample), and dashed lines show the cumulative 0.05 and 0.95 quantiles. Red circles are the MPD estimates.}{Central Run 75: }{ymr.}
%
%\clearpage
%
%\twofig{boverbmsyMCMC}{depleteMCMC}{Top: estimated spawning biomass $B_t$ relative to spawning biomass at maximum sustainable yield ($\Bmsy$) (boxplots). The median biomass trajectory appears as a solid curve surrounded by a 90\pc{} credibility envelope (quantiles: 0.05-0.95) in light blue and delimited by dashed lines for years $t$=1935-2021; projected biomass appears in light red for years $t$=2022-2031. Also delimited is the 50\pc{} credibility interval (quantiles: 0.25-0.75) delimited by dotted lines. The horizontal dashed lines show the median LRP and USR. Bottom: marginal posterior distribution of depletion ($B_t/B_0$), where $t$=1935-2021.}{Central Run 75: }{ymr.}
%
%
%
%\onefig{snail}{Phase plot through time of the medians of the ratios $B_t/\Bmsy$ (the spawning biomass in year $t$ relative to $\Bmsy$) and $u_{t-1} / \umsy$ (the exploitation rate in year $t-1$ relative to $\umsy$). The filled green circle is the starting year (1936). Years then proceed from light grey through to dark grey with the final year (2021) as a filled cyan circle, and the blue lines represent the 0.05 and 0.95 quantiles of the posterior distributions for the final year. The filled gold circle indicates the status in 2011, which coincides with a previous assessment for this species. Red and green vertical dashed lines indicate the Precautionary Approach provisional limit and upper stock reference points (0.4, 0.8 $\Bmsy$), and the horizontal grey dotted line indicates $u$ at RMD.}{Central Run 75: }{ymr.}
%
%\clearpage
%
%\twofig{sbiomassMCMC}{sprMCMC}{Marginal posterior distribution of spawning biomass (top) and spawners-per recruit (bottom) over time. Boxplots show the 0.05, 0.25, 0.5, 0.75, and 0.95 quantiles from the MCMC results.}{Central Run 75: }{ymr.}
%
%\twofig{recruitsMCMC}{recdevMCMC}{Marginal posterior distribution of recruitment in 1,000s of age-0 fish (top) and recruitment deviations (bottom) over time. Boxplots show the 0.05, 0.25, 0.5, 0.75, and 0.95 quantiles from the MCMC results.}{Central Run 75: }{ymr.}
%
%\twofig{fishmortMCMC}{exploitMCMC}{Marginal posterior distribution of fishing mortality (top) and exploitation rate (bottom) over time. Boxplots show the 0.05, 0.25, 0.5, 0.75, and 0.95 quantiles from the MCMC results.}{Central Run 75: }{ymr.}
%
%\clearpage
%
%%==============================================================================
