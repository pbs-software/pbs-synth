%%Annexe F R\'{e}sultats -- Tableaux et figures relatives au MDP (mode de la densit\'{e} a posteriori) du cycle central -- commence \`{a} la page 187 du document de r\'{e}f\'{e}rence

%%\newpage
\subsubsection{Tableaux relatifs au MDP du cycle central}

%%---Table 2-----------------------------
\setlength{\tabcolsep}{4pt}
\begin{table}[!h]
\centering
\caption{Cycle central E75~: Valeurs a priori et estimations du MDP pour les param\`{e}tres estim\'{e}s. Information a priori -- distributions~: 0~=~uniforme, 2~=~b\^{e}ta, 6~=~normale}
\label{tab:ymr.parest}
\usefont{\encodingdefault}{\familydefault}{\seriesdefault}{\shapedefault}\small
\begin{tabular}{lcccccr}
\hline \\ [-1.5ex]
%\multicolumn{6}{l}{{\bf Parameter in write-up, Awatea input name, Awatea export name}} \\
{\bf Param\`{e}tre} & {\bf Phase} & {\bf Plage} & {\bf Type} & {\bf (Moyenne, \'{e}cart-type)} & {\bf Initiale} & {\bf MDP} \\ [1ex]
\hline \\ [-1.5ex]
LN (R0) & 1 & (1, 16) & 6 & (8, 8) & 8 & 8,062 \\
mu (1) CHALUT+ & 3 & (1, 40) & 6 & (10,7, 2,14) & 10,7 & 11,645 \\
varL(1) CHALUT+ & 4 & (-15, 15) & 6 & (1,6, 0,32) & 1,6 & 2,073 \\
mu (2) BRC & 3 & (1, 40) & 6 & (15,6, 3,12) & 15,6 & 13,599 \\
varL (2) BRC & 4 & (-15, 15) & 6 & (3,72, 0,744) & 3,72 & 3,915 \\
mu (3) COIV & 3 & (1, 40) & 6 & (15,4, 3,08) & 15,4 & 13,738 \\
varL (3) COIV & 4 & (-15, 15) & 6 & (3,44, 0,688) & 3,44 & 3,820 \\
mu (4) COHG & 3 & (1, 40) & 6 & (10,8, 2,16) & 10,8 & 10,834 \\
varL (4) COHG & 4 & (-15, 15) & 6 & (2,08, 0,416) & 2,08 & 2,017 \\
mu (5) GIG & 3 & (1, 40) & 6 & (17,4, 3,48) & 17,4 & 15,753 \\
varL (5) GIG & 4 & (-15, 15) & 6 & (4,6, 0,92) & 4,6 & 4,828 \\
\hline
\end{tabular}
\usefont{\encodingdefault}{\familydefault}{\seriesdefault}{\shapedefault}\normalsize
\end{table}

\newpage
\subsubsection{Figures du MDP du cycle central}

\onefig{mleParameters}{Profils de vraisemblance (courbes bleues fines) et fonctions de la densit\'{e} a priori (courbes noires \'{e}paisses) des param\`{e}tres estim\'{e}s. Les lignes verticales repr\'{e}sentent les estimations du maximum de vraisemblance; les triangles rouges indiquent les valeurs initiales utilis\'{e}es dans le processus de minimisation.}{Cycle central E75~: }{ymr.}

\onefig{survIndSer}{Valeurs de l'indice de relev\'{e} (points) avec intervalles de confiance \`{a} 95\pc{} (barres) et ajustements du mod\`{e}le au MPD (courbes) de la s\'{e}rie des relev\'{e}s ind\'{e}pendants de la p\^{e}che.}{Cycle central E75~: }{ymr.}

\clearpage

\onefig{agefitFleet1}{Proportions de la p\^{e}che chalut+ selon l'\^{a}ge (barres = observ\'{e}es, lignes = pr\'{e}vues) pour les femelles et les m\^{a}les r\'{e}unis.}{Cycle central E75~: }{ymr.}
\onefig{ageresFleet1}{R\'{e}sidus du mod\`{e}le de la p\^{e}che chalut+ par rapport aux donn\'{e}es sur la proportion selon l'\^{a}ge. Les axes verticaux correspondent aux r\'{e}sidus normalis\'{e}s. Les diagrammes de quartiles dans les trois panneaux montrent les r\'{e}sidus par classe d'\^{a}ge, par ann\'{e}e de donn\'{e}es et par ann\'{e}e de naissance (en suivant une cohorte dans le temps). Les bo\^{i}tes de la cohorte sont en vert si les \'{e}carts de recrutement au cours de l'ann\'{e}e de naissance sont positifs, en rouge s'ils sont n\'{e}gatifs. Elles fournissent des plages de quantiles (0,25-0,75) munies d'une ligne horizontale trac\'{e}e au niveau de la m\'{e}diane, des moustaches verticales s'\'{e}tendent jusqu'aux quantiles 0,05 et 0,95, et des valeurs aberrantes apparaissent sous forme de signes plus.}{Cycle central E75~: }{ymr.}
\clearpage

\onefig{agefitFleet2}{Relev\'{e} synoptique dans le bassin de la Reine-Charlotte (BRC) -- proportions selon l'\^{a}ge (barres = observ\'{e}es, lignes = pr\'{e}vues) pour les femelles et les m\^{a}les r\'{e}unis.}{Cycle central E75~: }{ymr.}
\onefig{ageresFleet2}{Relev\'{e} synoptique dans le BRC -- r\'{e}sidus des ajustements du mod\`{e}le aux donn\'{e}es sur la proportion selon l'\^{a}ge. Se reporter \`{a} la l\'{e}gende de la Fig.~\ref{fig:ymr.ageresFleet1} afin d'obtenir les d\'{e}tails sur le graphique.}{Cycle central E75~: }{ymr.}
\clearpage

\onefig{agefitFleet3}{Relev\'{e} synoptique sur la c\^{o}te ouest de l'\^{i}le de Vancouver (COIV) -- proportions selon l'\^{a}ge (barres = observ\'{e}es, lignes = pr\'{e}vues) pour les femelles et les m\^{a}les r\'{e}unis.}{Cycle central E75~: }{ymr.}
\onefig{ageresFleet3}{Relev\'{e} synoptique sur la COIV -- r\'{e}sidus des ajustements du mod\`{e}le aux donn\'{e}es sur la proportion selon l'\^{a}ge. Se reporter \`{a} la l\'{e}gende de la Fig.~\ref{fig:ymr.ageresFleet1} afin d'obtenir les d\'{e}tails sur le graphique.}{Cycle central E75~: }{ymr.}
\clearpage

\onefig{agefitFleet4}{Relev\'{e} synoptique sur la c\^{o}te ouest de Haida Gwaii (COHG) -- proportions selon l'\^{a}ge (barres = observ\'{e}es, lignes = pr\'{e}vues) pour les femelles et les m\^{a}les r\'{e}unis.}{Cycle central E75~: }{ymr.}
\onefig{ageresFleet4}{Relev\'{e} synoptique sur la COHG -- r\'{e}sidus des ajustements du mod\`{e}le aux donn\'{e}es sur la proportion selon l'\^{a}ge. Se reporter \`{a} la l\'{e}gende de la Fig.~\ref{fig:ymr.ageresFleet1} afin d'obtenir les d\'{e}tails sur le graphique.}{Cycle central E75~: }{ymr.}
\clearpage

\onefig{agefitFleet5}{Relev\'{e} historique dans le goulet de l'\^{i}le Goose (GIG) -- proportions selon l'\^{a}ge (barres = observ\'{e}es, lignes = pr\'{e}vues) pour les femelles et les m\^{a}les r\'{e}unis.}{Cycle central E75~: }{ymr.}
\onefig{ageresFleet5}{Relev\'{e} historique dans le GIG -- r\'{e}sidus des ajustements du mod\`{e}le aux donn\'{e}es sur la proportion selon l'\^{a}ge. Se reporter \`{a} la l\'{e}gende de la Fig.~\ref{fig:ymr.ageresFleet1} afin d'obtenir les d\'{e}tails sur le graphique.}{Cycle central E75~: }{ymr.}
\clearpage

\onefig{harmonica0}{Moyenne harmonique de la taille effective de l'\'{e}chantillon (ligne pointill\'{e}e horizontale) par rapport \`{a} la moyenne arithm\'{e}tique de la taille observ\'{e}e de l'\'{e}chantillon (ligne pointill\'{e}e verticale) pour les fr\'{e}quences d'\^{a}ge non pond\'{e}r\'{e}es. La ligne pleine montre un rapport 1:1; la courbe en pointill\'{e} rouge d\'{e}signe un ajustement par r\'{e}gression polynomiale de premier ordre.}{Cycle central E75~: }{ymr.}

%\onefig{harmonica}{\textbf{Weighted AF} -- Moyenne harmonique de la taille effective de l'\'{e}chantillon par rapport \`{a} la moyenne arithm\'{e}tique de la taille ajust\'{e}e de l'\'{e}chantillon.}{Cycle central E75~: }{ymr.}

\clearpage

%\onefig{meanAge0}{\^{A}ges moyens calcul\'{e}s chaque ann\'{e}e pour les donn\'{e}es \textbf{non pond\'{e}r\'{e}es} (cercles pleins) avec des intervalles de confiance \`{a} 95\pc{} et des estimations par le mod\`{e}le (lignes bleues) pour les donn\'{e}es sur l'\^{a}ge tir\'{e}es de la p\^{e}che commerciale et des relev\'{e}s.}{Cycle central E75~: }{ymr.}

\onefig{meanAge}{\^{A}ges moyens calcul\'{e}s chaque ann\'{e}e pour les donn\'{e}es pond\'{e}r\'{e}es (cercles verts pleins), o\`{u} les barres verticales indiquent la plage des donn\'{e}es et les barres transversales repr\'{e}sentent les intervalles de confiance \`{a} $\sim$95\pc{} associ\'{e}s aux tailles d'\'{e}chantillon ajust\'{e}es; les estimations du mod\`{e}le de l'\^{a}ge moyen apparaissent sous forme de lignes bleues.}{Cycle central E75~: }{ymr.}

\clearpage

\onefig{selectivity}{S\'{e}lectivit\'{e}s pour les prises de la flotte commerciale et les relev\'{e}s (toutes les valeurs du MDP), o\`{u} la courbe de maturit\'{e} pour les femelles est form\'{e}e par les lettres `m'.}{Cycle central E75~: }{ymr.}

\twofig{Bt}{BtB0}{Biomasse f\'{e}conde -- (en haut) $B_t$ (tonnes, femelles matures) au fil du temps; (en bas) $B_t$ par rapport \`{a} la biomasse f\'{e}conde \`{a} l'\'{e}quilibre non exploit\'{e}e $B_0$. La ligne bleue correspond \`{a} l'ajustement de la plateforme Stock Synthesis (SS) pour \currYear.}{Cycle central E75~: }{ymr.}

%%\twofig{recruits}{recDev}{Recrutement (milliers de poissons) au fil du temps (en haut) et consignation des \'{e}carts de recrutement annuels (en bas), $\epsilon_t$, o\`{u} l'\'{e}cart multiplicatif corrig\'{e} en fonction du biais est $\mbox{e}^{\epsilon_t - \sigma_R^2/2}$ et $\epsilon_t \sim \mbox{Normal}(0, \sigma_R^2)$. La ligne bleue correspond \`{a} l'ajustement de la plateforme SS pour les poissons d'\^{a}ge 0 en \currYear{}.}{Cycle central E75~: }{ymr.}

%% #1 = filename 1 & label, #2 = filename 2, #3 = filename 3, #4 = caption, #5=caption prefix (optional), #6=label prefix (optional)
\threefig{recruits}{recDev}{stockRecruit}{Recrutement -- (en haut) milliers de poissons d'\^{a}ge 0 entre \startYear{} et \currYear; (au milieu) consignation des \'{e}carts de recrutement annuels $\epsilon_t$, o\`{u} l'\'{e}cart multiplicatif corrig\'{e} en fonction du biais est $\mbox{e}^{\epsilon_t - \sigma_R^2/2}$ et $\epsilon_t \sim \mbox{Normal}(0, \sigma_R^2)$; (en bas) relation stock-recrutement d\'{e}terministe (courbe noire) et valeurs observ\'{e}es (identifi\'{e}es par l'ann\'{e}e de fraie).}{Cycle central E75~: }{ymr.}

\clearpage
%%==============================================================================
